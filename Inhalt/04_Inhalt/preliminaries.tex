\chapter{Preliminaries}

	In order get a better intuition about the relevant complexity landscape, PSPACE-Completeness and the gadget construction in particular, which will become very important in section \ref{chap:gadgets}, a few selected topics are briefly presented on the following pages.
	Readers which already have a background in theoretical computer science or related fields may skip these sections, as they only contain well known results.

	\section{Complexity landscape}
	
		\begin{figure}[ht!]
			\centering
			\begin{tikzpicture}[myellipse/.style 2 args={ellipse, fill=black!#1, label={[anchor=north, below=2.5mm]#2}}, font=\sffamily]
				\node[myellipse={10}{EXPSPACE}, minimum width=12cm, minimum height=5cm, draw] (e1) {};
				\node[myellipse={20}{EXPTIME}, minimum width=10cm, minimum height=4cm, above=2mm of e1.south] (e2) {};
				%\node[myellipse={30}{PSPACE = NPSPACE}, minimum width=8cm, minimum height=4cm, above=2mm of e2.south] (e3) {};
				\node[myellipse={30}{PSPACE = NPSPACE}, minimum width=8cm, minimum height=3cm, above=2mm of e2.south] (e3) {};
				\node[myellipse={40}{NP}, minimum width=6cm, minimum height=2cm, above=2mm of e3.south] (e4) {};
				\node[myellipse={50}{P}, minimum width=4cm, minimum height=1cm, above=2mm of e4.south] (e5) {};
				%\node[myellipse={40}{}, minimum width=2cm, minimum height=1cm, above=2mm of e5.south] (e6) {NL};
			\end{tikzpicture}
			\caption{An overview about the complexity landscape relevant for this report. The class $\PSP$ is a superset of the known classes like $\PO$ and $\NP$. Just like the famous $\PO \stackrel{?}{=} \NP$ problem, it is still unknown whether $\PO \stackrel{?}{=} \PSP$. One interesting aspect, which is already indicated in the figure, is the fact that $\PSP=\NPSP$, which follows as a corollary from \textit{Savitch's theorem} \cite{Savitch.1970, Arora.2010}.}
			\label{fig:prelims:landscape}
		\end{figure}
		
		In the following sections the most important complexity classes relevant for this report are characterized \todo{Klingt scheiße} and described in a intuitive way.
		As shown in the well known complexity class hierarchy in figure \ref{fig:prelims:landscape}, classes like $\PO$ and $\NP$ are contained in $\PSP$. 
	
		\subsection{P, NP, PSPACE}
		
			\edef\sizetape{0.7cm}
			\tikzstyle{tmtape}=[draw,minimum size=\sizetape]
			\tikzstyle{tmhead}=[arrow box,draw,minimum size=.5cm,arrow box
			arrows={east:.25cm, west:0.25cm}]
		
			\begin{figure}[ht!]
				\centering
				\begin{tikzpicture}
					\tikzstyle{every path}=[very thick]
					
					\edef\sizetape{0.7cm}
					\tikzstyle{tmtape}=[draw,minimum size=\sizetape]
					\tikzstyle{tmhead}=[arrow box,draw,minimum size=.5cm,arrow box
					arrows={north:.25cm}]
					
					%% Draw TM tape
					\begin{scope}[start chain=1 going right,node distance=-0.15mm]
						\node [on chain=1,tmtape,draw=none] {$\ldots$};
						\node [on chain=1,tmtape] {\textvisiblespace};
						\node [on chain=1,tmtape] (input) {1};
						\node [on chain=1,tmtape] {0};
						\node [on chain=1,tmtape] {0};
						\node [on chain=1,tmtape] {0};
						\node [on chain=1,tmtape] {0};
						\node [on chain=1,tmtape] {0};
						\node [on chain=1,tmtape] {\textvisiblespace};
						\node [on chain=1,tmtape,draw=none] {$\ldots$};
						\node [on chain=1] {\textbf{Input/Output Tape}};
					\end{scope}
					
					%% Draw TM head below (input) tape cell
					\node [tmhead,yshift=-.60cm,label=0:\textbf{Current head position and state}] at (input.south) (head) {$q_1$};
				\end{tikzpicture}
				\caption{An arbitrary \textit{Turing Machine} in state $q_1$ with a single tape in order to keep the drawings simple. Turing machines are often depicted with multiple tapes \cite{Arora.2010}, but these TMs can be simulated by a TM with just one tape using only quadratically more computation time.}
				\label{fig:prelims:tm}
			\end{figure}
			
			\begin{defStrich}[PTIME]
				PTIME, often abbreviated with just \textit{P}, is a complexity class which contains all \textit{decision problems} that can be solved by a deterministic turing machine in polynomial time.
			\end{defStrich}
			
			With respect to figure \ref{fig:prelims:landscape}, this means that for any decision problem in \textbf{PTIME} there exists a turing machine with can be solved after polynomial many steps.
			
			\begin{defStrich}[NPTIME]
				NPTIME, often abbreviated with just \textit{NP}, is a complexity class which contains all \textit{decision problems} that can be solved by a \textbf{non}-deterministic turing machine in polynomial time.
			\end{defStrich}
			
			Another definition which is quite common is the notion of a deterministic turing machine called a \textit{verifier} which can verify a given \textit{certificate} of polynomial length in polynomial time.
		
		\subsection{The dynamics of PSPACE and NPSPACE}
		
		\subsection{The essence of PSPACE}
	
	\section{\ac{QBF}}
	
		\todo{QBF definieren}
	
		QBF can also be used as a framework to encode and solve a variety of seemingly unrelated problems, e.g. Model Checking \cite{Baier.2008?}, Games \cite{Ani.2022} and many more applications that are beyond the scope of this report.
		For a more sophisticated overview about possible applications, the reader is referred to \cite{Shukla.2019}.
		In fact, even with very strong restrictions on the amount and type of quantifiers, these special variants of \ac{QBF} can still be used to encode many problems \cite{Balabanov.2016}.
	
	\section{Gadget Constructions}
	
	\section{0/1/2-Player Games}